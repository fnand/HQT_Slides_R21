%%%%%%%%%%%%%%%%%%%%%%%%%%%%%%%%%%%%%%%%%
% Beamer Presentation
% LaTeX Template
% Version 1.0 (10/11/12)
%
% This template has been downloaded from:
% http://www.LaTeXTemplates.com
%
% License:
% CC BY-NC-SA 3.0 (http://creativecommons.org/licenses/by-nc-sa/3.0/)
%
%%%%%%%%%%%%%%%%%%%%%%%%%%%%%%%%%%%%%%%%%

%----------------------------------------------------------------------------------------
%	PACKAGES AND THEMES
%----------------------------------------------------------------------------------------

\documentclass{beamer}

\mode<presentation> {

% The Beamer class comes with a number of default slide themes
% which change the colors and layouts of slides. Below this is a list
% of all the themes, uncomment each in turn to see what they look like.

%\usetheme{default}
%\usetheme{AnnArbor}
%\usetheme{Antibes}
%\usetheme{Bergen}
%\usetheme{Berkeley}
\usetheme{Berlin}
%\usetheme{Boadilla}
%\usetheme{CambridgeUS}
%\usetheme{Copenhagen}
%\usetheme{Darmstadt}
%\usetheme{Dresden}
%\usetheme{Frankfurt}
%\usetheme{Goettingen}
%\usetheme{Hannover}
%\usetheme{Ilmenau}
%\usetheme{JuanLesPins}
%\usetheme{Luebeck}
%\usetheme{Madrid}
%\usetheme{Malmoe}
%\usetheme{Marburg}
%\usetheme{Montpellier}
%\usetheme{PaloAlto}
%\usetheme{Pittsburgh}
%\usetheme{Rochester}
%\usetheme{Singapore}
%\usetheme{Szeged}
%\usetheme{Warsaw}

% As well as themes, the Beamer class has a number of color themes
% for any slide theme. Uncomment each of these in turn to see how it
% changes the colors of your current slide theme.

%\usecolortheme{albatross}
%\usecolortheme{beaver}
%\usecolortheme{beetle}
%\usecolortheme{crane}
%\usecolortheme{dolphin}
%\usecolortheme{dove}
%\usecolortheme{fly}
%\usecolortheme{lily}
%\usecolortheme{orchid}
%\usecolortheme{rose}
%\usecolortheme{seagull}
%\usecolortheme{seahorse}
%\usecolortheme{whale}
%\usecolortheme{wolverine}

%\setbeamertemplate{footline} % To remove the footer line in all slides uncomment this line
%\setbeamertemplate{footline}[page number] % To replace the footer line in all slides with a simple slide count uncomment this line

%\setbeamertemplate{navigation symbols}{} % To remove the navigation symbols from the bottom of all slides uncomment this line
}

\usepackage{graphicx} % Allows including images
\usepackage{booktabs} % Allows the use of \toprule, \midrule and \bottomrule in tables
\usepackage{enumitem}


\usefonttheme{serif}
\usepackage{mathptmx}
\usepackage[scaled=0.9]{helvet}
\usepackage{courier}

\usepackage{listings}
\lstset{
basicstyle=\small\ttfamily,
columns=flexible,
breaklines=true
}
%----------------------------------------------------------------------------------------
%	TITLE PAGE
%----------------------------------------------------------------------------------------

\title[HQT Meeting]{R21 VLQ SP Signal Production Proposal} % The short title appears at the bottom of every slide, the full title is only on the title page

\author{Ferdinand Schenck, Didier Alexandre} % Your name
\institute[HU-Berlin] % Your institution as it will appear on the bottom of every slide, may be shorthand to save space
{
%HU-Berlin \\ % Your institution for the title page
\medskip
\textit{ferdinand.schenck@cern.ch} % Your email address
}
\date{\today} % Date, can be changed to a custom date

\begin{document}

\begin{frame}
\titlepage % Print the title page as the first slide
\end{frame}

%\begin{frame}
%\frametitle{Overview} % Table of contents slide, comment this block out to remove it
%\tableofcontents % Throughout your presentation, if you choose to use \section{} and \subsection{} commands, these will automatically be printed on this slide as an overview of your presentation
%\end{frame}

%----------------------------------------------------------------------------------------
%	PRESENTATION SLIDES
%----------------------------------------------------------------------------------------

%------------------------------------------------
\section{Signal Samples} % Sections can be created in order to organize your presentation into discrete blocks, all sections and subsections are automatically printed in the table of contents as an overview of the talk
%------------------------------------------------


\subsection{Rel 20.7} % A subsection can be created just before a set of slides with a common theme to further break down your presentation into chunks



\begin{frame}
\frametitle{The Problem}
Release 20.7 signal samples did not include interference with SM. 
\linebreak
\linebreak
As we have realised in the current analysis interference might have a significant effect in Single Production VLQ to Wb. 
\linebreak
\linebreak
The interfering background is not included in current backgound MC.




\end{frame}


%------------------------------------------------------------------------%
\begin{frame}
\frametitle{Our Proposal}
Our proposal is to produce samples similar to before, plus additional samples containing interference with SM and SM background. 
\linebreak
\linebreak
Thus 3 Type of samples:
\end{frame}

%------------------------------------------------------------------------%


\begin{frame}
\frametitle{Samples}
\begin{itemize}[topsep=1cm]


\item{*} 1 File (total) with interfering SM background
\item{*} 1 File per mass point for VLQ signal
\item{*} 2 Files per mass point for interference (one for T, one for Y)
\end{itemize}

This gives us the flexibility to scale SM, signal and interference seperately.
\end{frame}

%------------------------------------------------------------------------%


\end{document} 